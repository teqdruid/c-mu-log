%  Written for Columbia COMS4112, fall 2008
\documentclass[onecolumn,titlepage]{article}
\usepackage[pdftex]{graphicx}
\usepackage{fullpage}
\usepackage{moreverb}
\usepackage{multicol}

\usepackage[left=2cm,top=2cm,bottom=2cm,right=2cm]{geometry}

\def\sourcetabsize{4}
\newenvironment{sourcestyle}{\begin{scriptsize}}{\end{scriptsize}}
\def\sourceinput#1{\par\begin{sourcestyle}\verbatimtabinput[\sourcetabsize]{#1}\end{sourcestyle}\par}

\usepackage{float}
 
\floatstyle{ruled}
\newfloat{program}{thp}{lop}
\floatname{program}{Program}


\begin{document}
\title{C$\mu$LOG Reference Manual\\\small{An Entity Interaction Simulation Language}}
\author{John Demme (jdd2127)\\Nishant Shah (nrs2127)\\Devesh Dedhia (ddd2121)\\Cheng Cheng (cc2999)
\\ \\Columbia University}

\maketitle

\section{Overview}
C$\mu$LOG is a logic language designed for entity interaction simulation.  It uses a syntax similar to C, making it easier on the typical programmer's eyes, and already integrates with some code tools, such as code indenters.  One uses the language to provide a set of facts and rules, and the ``program'' is run by asking a question, which the interpreter attempts to answer using inference based on the fact and rule set.  C$\mu$LOG is designed for simulation, so typically a simulator will ask a given agent program what it's next action will be.  The agent program then uses C$\mu$LOG's entity interaction features to gather information about it's environment and decide what to do.

\section{Goals}
The language presented here attempts to fulfill the following requirements:
\begin{itemize}
\item Generic- games are defined completely by the environment application.

\item Composable- individual behaviors can be written simply and easily, then combined
  to obtain high-level actions and reasoning.

\item Declarative- programmers can specify what they want entities to do rather than how

\item Controlled Communication- data in the system is frequently made up of nearly-atomic bits of data
  many of which can be used both on their own and composed as complex data.  This means that subsets
  and smaller pieces of data can be communicated between entities without loosing meaning.

\item High-level libraries- due to the flexibility of the language, high-level
  algorithms--such as path-finding--can be easily implemented in libraries, allowing
  further, domain-specific intelligence to be written in the programs.
\end{itemize}

\section{Lexical}
\begin{verbatim}
[' ' '\t' '\r' '\n'] WS
"/*"     OPENCOMMENT
"*/"     CLOSECOMMENT
"//"     COMMENT
'('      LPAREN
')'      RPAREN
'{'      LBRACE
'}'      RBRACE
';'      SEMICOLON
','      COMMA
'+'      PLUS
'-'      MINUS
'*'      TIMES
'/'      DIVIDE
"=="     EQ
"!="     NEQ
'<'      LT
"<="     LEQ
">"      GT
">="     GEQ
'@'      AT
'.'	 DOT
'['      ARROPEN
']'      ARRCLOSE
'"'      QUOTE
'$'['a'-'z' 'A'-'Z']['a'-'z' 'A'-'Z' '0'-'9' '_']* Variable
['0'-'9']+ Number
['a'-'z' 'A'-'Z']['a'-'z' 'A'-'Z' '0'-'9' '_']* Identifier
\end{verbatim}

\section{Facts}
Facts define factual relationships.  They have a very similar syntax to rules, except they
have no code block to make them conditionally true.  Any query which matches a fact is simply
true. Another way to think of facts is as terminal nodes in the solution search.

Each fact is composed of a name, and a comma separated list of parameters, each of which may
be a constant, or a variable.  Using any variable except the anonymous variable doesn't make
much sense in a fact, but is allowable.
\begin{verbatim}
Example:
  foo(4, symA);  //Foo of 4 and symA is always true
  foo(4, symA, ?); //Foo of 4, symA, and anything (wildcard) is always true
  wall(4, 5); //In an environment might mean: there is a wall present at (4,5)

Grammar:
  Fact -> Identifier ( ParamList );
  ParamList -> Param | ParamList , Param
  Param -> Variable | Number | String | Array
\end{verbatim}

\section{Rules}
Rules define relationships which are conditionally true.  They are similar to facts, but
instead of ending with a semicolon, they contain have a block, which defines the conditions
upon which the rule should be evaluated as true.  Another way to think of a rules is as
a node in the solution search which may branch, or be a leaf, depending on the contents
of the condition block. Each rule is composed of a name, a comma separated list of parameters,
and a block.

\begin{verbatim}
Example:
  foo(4) { bar(5); }  //Foo of 4 is true if bar(5) is true
  foo(4) { bar(6); }  //Foo of 4 is true if bar(6) is true
The two above rules are together equivalent to:
  foo(4) {OR: bar(5); bar(6); }

Grammar:
  Fact -> Identifier ( ParamList ) Block
\end{verbatim}

\section{Variables}
Variables represent a value to be solved for.  During rule matching, they will match any
value or type, but can be constrained in an associated block.  All variables are scoped to
the rule, so that variable solutions can be shared between subblocks.
Variables are represented by a dollar sign (\$) then the variable name. The name must
start with a letter, and is composed of letters, numbers, and underscores.  There is a special
variable called the anonymous variable which is represented simply by a question mark (?).  It
cannot be referenced in the block, and simply matches anything.
\begin{verbatim}
Example:
  foo($X, $y, $foo_bar, $bar9, ?) { }

Grammar:
  Variable -> $[a-zA-Z][a-zA-Z0-9_]* | ?
\end{verbatim}

\section{Blocks}
Blocks contain a list of statements (conditions) to determine truth, and specify a reduction 
method for those statements.  Each block will reduce all of its statements using the same
reduction method (usually AND or OR), but may contain sub-blocks.  If the reduction method
is omitted, AND is assumed.  The syntax allows for other reduction methods to be allowed
(such as xor, or a user-specified method), however the language does not yet support this.
\begin{verbatim}
Examples:
 { 
    foo();
    bar();
 }
 //True if foo and bar are both true.

 {AND:
    foo();
    bar();
 }
 //True if foo and bar are both true.

 {OR:
    foo();
    bar();
 }
 //True if foo or bar are true.

Grammar:
  Block -> { (Identifer:)? StatementList }
  StatementList -> Statement | StatementList Statement
\end{verbatim}

\section{Statements}
Statements are boolean qualifiers which are used inside of blocks.  They can be any one
of three types: comparisons, evaluations, or blocks. Comparisons are used to constrain variables.
N-ary comparisons are supported.  Only values of the same type can be compared, and certain
comparisons only work on certain types, so comparisons can be used to constrain variables by 
type.  Evals are used to query the program, and have a similar syntax as facts.  They can be thought
of as a branch in the solution search.  Blocks are considered a statement to support sub-blocks.
They are evaluated and the reduced result is used.  Comparisons and evals are both terminated
by semicolons.

\begin{verbatim}
Examples:
  1 < $X <= $Y < 10;  // A comparison
  range($X, $Y, 7);   // An eval
  !range($X, $Y, 7);   // This must not evaluate to true
  {OR: $X > 10; $X < 0; }  //A sub-block with two binary comparisons

Grammar:
  Statement -> Block | Eval ; | Comparison ;
  Eval -> (!)? Identifier ( ExprList );
  ExprList -> Expression | ExprList , Expression
  Comparison -> Expression ComparisonOp Expression | Expression ComparisonOp Comparison
  ComparisonOp -> EQ | NEQ | LT | LEQ | GT | GEQ
\end{verbatim}

\section{Expressions}
Expressions are used to modify values being passed into comparisons or evals.  They are
used to modify integers, and supports plus, minus, times, and divide.  Typical infix
notation and precedence rules are used, and expressions can be grouped with parenthesis.

\begin{verbatim}
Examples:
  $r - 10 < $X < $r + 10;  // A comparison: $r - 10 and $r + 10 are the expressions
  range($X, $Y, $r / 3);   // An eval- $r / 3 is the expression here

Grammar:
  Expression -> Number | String | Variable | Expression Op Expression | ( Expression )
  Op -> PLUS | MINUS | TIMES | DIVIDE
\end{verbatim}

\section{Types}
The following types are supported: integers, strings, arrays, symbols, and entities.
Strings in C$\mu$LOG are currently atomic, so no string processing such as splitting,
joining, or searching is supported.  They are primarily used for interaction with the
rest of the system (printing, specifying files, ect.).  Arrays are discussed in detail
in the next section.  Symbols are simply identifiers.  They share the same namespace as rule
and fact names, and can only be compared with equals and not equals.  Entities are used
to represent other programs (typically agents) and are used for interaction.  In addition
to equals and not equals comparison operators, they support the dot operator for interaction
(discussed later.)

\section{Arrays}
Arrays in C$\mu$LOG behave similarly to functional lists, and are matched similiarly as well.
They are delimited by [ and ], and elements are comma separated.  The null list (also the tail
of the list) is denoted by [].  Any number of elements of the array are matched inside of
fact and rule declarations.  Arrays only appear as fact and rule parameters.
The last element listed in [] is an array, and represents the tail of the list.  If the
elements listed before the last element comprise the entire list, the last element will be
simply [].

\begin{verbatim}
Examples:
  //A set of rules which ensures that all elements are less than $v
  lessThan($v, [$head, $tail]) { $head < $v; lessThan($v, $tail); }
  lessThan(?, []);

  //Matches the list (1, 4, 6)
  foo([1, 4, 6, []]);

  //Invalid
  foo([1, 4, 6]);

  //Prepends $v to array $a
  prepend($v, $a, [$v, $a]);

  //Constructs a list
  list($a, $b, [$a, $b, []]);

Grammar:
  Array -> [ ArrayElems ]
  ArrayElems -> ArrayElem | ArrayElem ArrayElems
  ArrayElem -> Number | String | Variable | []
\end{verbatim}

\section{Directives}
C$\mu$LOG supports a special syntax for interpreter directives.  This allows programs to
interact with the interpreter while avoiding symbol collisions. The syntax is similar
to that of a fact's, but an at sign (@) is prepended. Four directives are currently planned:
attach, print, learn, and forget.  Attach is used to include code from another C$\mu$LOG file.
Print is used to output strings, and results of searches during runtime.  Learn and forget
are discussed in the next section.
\begin{verbatim}
Examples:
  @attach("geometry.ul");
  @print("Hello, world!");

Grammar:
  Directive -> @ Identifier ( ParamList ); 
\end{verbatim}

\section{Program Modification}
The two directives learn and forget are used to modify a program at runtime.  This
is the only way in which C$\mu$LOG supports non-volatile storage.  Learn is used to
add a fact to a program, and forget is used to remove a fact.  The synatax for these
two directives is special, consisting of the usual directive syntax, except contained
inside the parenthesis is a fact definition.  Any non-anonymous variables in this fact
definition are filled in with solutions found for those variables, and the learn or
forget is ``executed'' once for each solution.  They are similar to Prolog's assert and retract.
\begin{verbatim}
Examples:
  @learn( wall(4,5); );  //Remember that there is a wall at (4,5)
  @forget( agent(8, 10); ); //Forget about the agent at (8, 10)

Grammar:
  Directive -> @ (learn|forget) ( Fact ); 
\end{verbatim}


\section{Interaction- The Dot Operator}
If a variable or symbol represents another program (entity), then it supports
the dot operator.  After appending a dot (.) to the reference, one can put
an eval, a learn, or a forget, and that action will take place in the other
entity's namespace.  This can be used to ask for information from another program
(such as the environment program or another agent) or to modify the other
program--perhaps to teach another agent, to trick a competitor, or to change
the operating environment.  Future versions of C$\mu$LOG could likely support
some sort of access rules in the destination program, allowing it to control who is
allowed to access what data, and who is allowed to change its program, and how.
These access rules could potentially modify any queries or changes, perhaps
revealing an entirely fake namespace to the other agent.  Such access rules are
beyond the scope of C$\mu$LOG initially, however.
\begin{verbatim}
Example:
  $agent.@learn( wall(4,5); );  //Tell agent2 that there is a wall at (4,5)
  env.view($X, $Y, $obj); //Query the environment, find out what is at ($X, $Y)

Grammar:
  DotOp -> Directive | Statement
  Dot -> Variable . DotOp | Identifier . DotOp
\end{verbatim}

\section{Example Code}
Several examples are now given.  They are not complete, and only intended
to give a gist of the language's syntax and semantics.

The environ1.ul program defines a 15x15 grid as well as several wall locutions.
The simulator doesn't know anything about a ``wall'' or a ``goal'', but gets
symbols for each grid point by solving the ``object'' rule.  The environment
interacts with the simulator primarily via the object rule.  The ``repr'' rule
is also used to tell the simulator what file should be associated with each
symbol.  This file could be an image (to display on the grid) or an agent program
to run, starting in that grid.

The agent1.ul program is a pretty simple program which attempts to reach the ``goalObject''
without running into anything else. To do this, it uses very simple graph-search 
algorithms to find a valid path to the goalObject, or--alternatively--a grid square
which it has not been to yet.  This sort of searching algorithm is very simple in
C$\mu$LOG due to it's logical nature.  Indeed, it depends on the simple search 
which is used internally to solve programs.  The agent also attempts to communicate it's 
knowledge of the environment to any other agents it encounters.


\begin{program}
\begin{multicols}{2}
\sourceinput{samples/environ1.ul}
\end{multicols}
\caption{A sample C$\mu$LOG environment programming}
\end{program}

\begin{program}
\begin{multicols}{2}
\sourceinput{samples/agent1.ul}
\end{multicols}
\caption{A sample C$\mu$LOG agent}
\end{program}

\end{document}
