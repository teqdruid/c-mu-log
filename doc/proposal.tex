%  Written for Columbia COMS4112, fall 2008
\documentclass[onecolumn,titlepage]{article}
\usepackage[pdftex]{graphicx}
\usepackage{fullpage}
\usepackage{moreverb}
\usepackage{multicol}

\usepackage[left=2cm,top=2cm,bottom=2cm,right=2cm]{geometry}

\def\sourcetabsize{4}
\newenvironment{sourcestyle}{\begin{scriptsize}}{\end{scriptsize}}
\def\sourceinput#1{\par\begin{sourcestyle}\verbatimtabinput[\sourcetabsize]{#1}\end{sourcestyle}\par}

\usepackage{float}
 
\floatstyle{ruled}
\newfloat{program}{thp}{lop}
\floatname{program}{Program}


%\acmVolume{V}
%\acmNumber{N}
%\acmYear{YY}
%\acmMonth{M}

\begin{document}
\title{C$\mu$LOG\\\small{An Entity Interaction Simulation Language}}
\author{John Demme (jdd2127)\\Nishant Shah (nrs2127)\\Devesh Dedhia (ddd2121)\\Cheng Cheng (cc2999)
\\ \\Columbia University}

\maketitle

\section{Overview}
C$\mu$LOG is a logic programming language and it resembles Prolog in many aspects.
The fundamental method of finding solutions is very similar to Prolog but C$\mu$LOG's
way of representing it is much more familiar to the end user. Additionally, C$\mu$LOG
has provisions for entity communication. As this language is going to be used for simulating 
real life motions, we
strongly emphasize that the program learn and forget data/rules/information at
run-time. For this, similar to ``assert'' and ``retract'' of Prolog we have
introduced two functions called ``learn'' and ``forget.''
  
In C$\mu$LOG there exist no specific data structures like you would see in Java or Python,
however rules and facts can be added to the program dynamically, which allows programs to
remember data in a much more natural way since the data simply becomes part of the running code.

The semantics of C$\mu$LOG are similar to Prolog, but the syntax is heavily modified to look 
more like C.  One of the most unique features about it is entity linking
and protection, which can be used to simulate entity communication.
To run a useful application on C$\mu$LOG you
will need to have an environment entity (defining the environment and game) and at least one agent,
 to provide a subject to be simulated. We link entities to the environment and the
access to data in all entities is dynamic-permission-based. Data access can help in an environment
when all agents have a common goal and data encapsulation can also be useful when the goals
are different.

Each agent program communicates with other programs to find out more information about
other agents or its own status in the environment. The simulator has all the information 
about each agent. This feature provides for interesting inter-program communication and at the same
time protects when necessary data. Examples of this are discussed in a subsequent section.

Another feature of C$\mu$LOG is that it has a powerful library of its own and also can
support third party libraries. The library supports functions like range, and move
which are very useful for simulating any environment. These functions are available in a
library and can also be complemented by the user to suit his needs. The default library 
will always be included in the program without any specific code but a third party
one would have to be explicitly included in the program in the following way:

@attach$<abc.xx>$


\section{Introduction to Logic Programming}
Logic programming is a kind of computer programming using
mathematical logic. Specifically, it is based on the the idea that
of applying a theorem-prover to declarative
sentences and deriving implications. Compared with procedural 
languages, logic programming solves the problem by
setting rules with which solutions must fit. We can represent
logic programming by the formula:
\begin{eqnarray*}
Facts + Rules = Solutions
\end{eqnarray*}

Logic programming languages provide several important advantages:
\begin{itemize}
\item Logic programming languages are inherently ``high level'' languages, allowing
  programmers to specify problems in a declarative manner, leaving some of all of the
  details of solving the problem to the interpreter.
\item The structures--both programming structures and data structures--in both
  prolog and C$\mu$LOG can be very simple- such as facts.  The
  relationship between code and data is also of note.  C$\mu$LOG uses the Von Neumann
  style (vs. Harvard architecture) wherein data is code.  It is therefore possible 
  (and inherently necessary) for programs to be introspective, and self-modifying.
  In other words, it is easier for programs to learn and adapt.
\end{itemize}
		
There are several different logic programming languages.
Among these languages, Prolog is the most popular one, which was developed
in 1972 by Alain Colmerauer. In prolog program, there are facts and
rules. The program is then driven by a question. It is widely used in 
artificial intelligence and computational linguistics.

\section{Applications}
The language C$\mu$LOG we have designed will be used for communication 
between agents and an environment, as well as to determine behavior of said
entities. Every agent program communicates 
with the environment program through a simulator. The simulator runs a
C$\mu$LOG logic solver and interpreter which functions on a set of rules 
defined and modified by the environment and agents then provides solutions
representing the actions to be taken by the agents.

The environment is grid based and defined by a C$\mu$LOG program.  It
potentially includes obstacles and a goals which the agent must reach, however
the game is defined almost entirely by the environment program.
Every object (i.e. agents, walls, switches, goals)
in the environment is defined by grid positions.  The environment specifies
the representations of the entities to the simulator.  The 
simulator re-evaluates the object rule during each turn when it renders
the grid, so the contents of the grid can be dynamically defined based on the state
of the simulation or the contents of the program (which can be changed by the program.)
 For example based on the grid position
of the agent the environment might remove or insert a wall. The agent program
decides the next move based on previous moves and obstacle data.

The simulation of the agent program is also turn based. Each time the agent 
makes a move it sends its new coordinates to the simulator. The new coordinates
become part of the simulation's rules which are exposed to the environment when
it is solved to render the scene.

The simulator discussed could be modified to be used for flight simulation
of several agents in a grid based airspace. Alternatively, it could be used in a real
environment like the movement of {\it pick and place} robots in a warehouse. 
The language could be used to define the warehouse environment and agent 
programs for robots.

\section{Example Code}
Several examples are now given.  They are not complete, and only intended
to give a gist of the language's syntax and semantics.

The environ1.ul program defines a 15x15 grid as well as several wall locutions.
The simulator doesn't know anything about a ``wall'' or a ``goal'', but gets
symbols for each grid point by solving the ``object'' rule.  The environment
interacts with the simulator primarily via the object rule.  The ``repr'' rule
is also used to tell the simulator what file should be associated with each
symbol.  This file could be an image (to display on the grid) or an agent program
to run, starting in that grid.

The agent1.ul program is a pretty simple program which attempts to reach the ``goalObject''
without running into anything else. To do this, it uses very simple graph-search 
algorithms to find a valid path to the goalObject, or--alternatively--a grid square
which it has not been to yet.  This sort of searching algorithm is very simple in
C$\mu$LOG due to it's logical nature.  Indeed, it depends on the simple search 
which is used internally to solve programs.  The agent also attempts to communicate it's 
knowledge of the environment to any other agents it encounters.

\section{Purpose}
The language proposed here is ideal for entity interaction and simulation for a 
number of important reasons:
\begin{itemize}
\item Generic- games are defined completely by the environment application.

\item Composable- individual behaviors can be written simply and easily, then combined
  to obtain high-level actions and reasoning.

\item Declarative- programmers can specify what they want entities to do rather than how

\item Controlled Communication- data in the system is frequently made up of nearly-atomic bits of data
  many of which can be used both on their own and composed as complex data.  This means that subsets
  and smaller pieces of data can be communicated between entities without loosing meaning.

\item High-level libraries- due to the flexibility of the language, high-level
  algorithms--such as path-finding--can be easily implemented in libraries, allowing
  further, domain-specific intelligence to be written in the programs.
\end{itemize}

\begin{program}
\begin{multicols}{2}
\sourceinput{samples/environ1.ul}
\end{multicols}
\caption{A sample C$\mu$LOG environment programming}
\end{program}

\begin{program}
\begin{multicols}{2}
\sourceinput{samples/agent1.ul}
\end{multicols}
\caption{A sample C$\mu$LOG agent}
\end{program}

\end{document}
